\documentclass[12pt]{article}
\usepackage{amssymb, amsmath, amsthm, harvard, datetime, setspace}
%\usepackage[pdftex]{graphicx}

\usepackage[a3paper,noheadfoot]{geometry}

% For a theorem, proposition etc. use \begin{thm} ... \end{thm} 
% Then use \begin{proof} ... \end{proof}
\theoremstyle{definition}
\swapnumbers
\newtheorem{thm}{Theorem}[section]
\newtheorem{propn}[thm]{Proposition}
\newtheorem{lem}[thm]{Lemma}
\newtheorem{defn}[thm]{Definition}
\newtheorem{cor}[thm]{Corollary}
\newtheorem{rem}[thm]{Remark}

\newcommand{\row}[1]{\mathrm{row}\,#1}
\newcommand{\s}[1]{\mathrm{sp}\,#1}
\newcommand{\R}{\mathbb{R}}
\newcommand{\im}{\textrm{im\,}}

\onehalfspacing


% For pictures: \pic{FileName}{Width in centimetres} 
%[The FileName must have a .jpg or .pdf in the Finder]
\newcommand{\pic}[3]{\begin{figure} \begin{center} \includegraphics[width= #2 cm]{#1} \caption{#3} \end{center} \end{figure}}

\textheight 15in
\textwidth 4in
\topmargin -.5in
\oddsidemargin 0pt
\evensidemargin 0pt 
\parskip = 0.0in
\parindent = 0.0in
\headheight = 0.0 in
\headsep = 0.0 in

\newtheorem{theorem}{Theorem}
\newtheorem{corollary}[theorem]{Corollary}
\newtheorem{definition}{Definition}



\title{\textbf{TITLE}}
\author{\textbf{AUTHOR}}
\date{\textbf{\shortdate\today}}


\begin{document}

% \begin{onehalfspace}
% \bibliographystyle{agsm}



We will show that the contrapositive of this statement is correct, i.e.~that if $n$ is not a multiple of 3, then $n^2$ is not a multiple of 3. So, as we are assuming that $n$ is not a multiple of 3, we know that $n=3k+1$ or $n=3k+2$ for some $k\in\mathbb{Z}$. We will consider each case in turn. Suppose $n=3k+1$, then $n^2=(3k+1)^2=9k^2+6k+1=3(3k^2+2k)+1$, which is not a multiple of 3. Now we consider the second case: $n^2=(3k+2)^2=9k^2+12k+4=3(3k^2+4k+1)+1$, which is not a multiple of 3. So, in both cases $n^2$ is not a multiple of 3, and we have established the contrapositive, which is equivalent to the original statement.



\newpage

\textbf{Theorem.} Let $f(x)$ be a real polynomial of degree $n\ge2$ with only real roots, such that $f(x)>0$ for $-1<x<1$ and $f(-1)=f(1)=0$. Let $A=\int_{-1}^1 f(x)\, dx$, and let $T$ be the area of the tangential triangle given by $f(x)$ (see figure). Then $\frac{2}{3}\cdot T\le A$.

\vspace{1cm}
Here is an argument about the claim:
\newpage
Since $f(x)$ has only real roots, and none of them in the open interval $(-1,1)$, it can be written --- apart from a constant positive factor which cancels out in the end --- in the form
\begin{equation}
f(x)=(1-x^2)\prod_i(\alpha_i-x)\prod_j(\beta_j+x)
\end{equation}
with $\alpha_i\ge1,\beta_j\ge1$. Hence
$$
A=\int_{-1}^{1}(1-x^2)\prod_i(\alpha_i-x)\prod_j(\beta_j+x)\,dx.
$$
By making the substitution $x\longmapsto -x$, we find that also
$$
A=\int_{-1}^{1}(1-x^2)\prod_i(\alpha_i+x)\prod_j(\beta_j-x)\,dx.
$$

and hence by the inequality of the arithmetic and geometric mean (note that all factors are $\ge0$)

\begin{eqnarray*}%{ccl}
A  & = &  \int_{-1}^{1} \frac{1}{2} [(1-x^2)\prod_i(\alpha_i-x)\prod_j(\beta_j+x) +(1-x^2)\prod_i(\alpha_i+x)\prod_j(\beta_j-x) ]\, dx\\
 & \ge & \int_{-1}^1 (1-x^2)\left(\prod_i(\alpha_i^2-x^2)\prod_j(\beta_j^2-x^2)\right)^\frac{1}{2}\, dx\\
 & \ge & \int_{-1}^1(1-x^2)\left(\prod_i(\alpha_i^2-1)\prod_j(\beta_j^2-1)\right)^\frac{1}{2}\, dx\\
 & = & \frac{4}{3}\left(\prod_i(\alpha_i^2-1)\prod_j(\beta_j^2-1)\right)^\frac{1}{2}
\end{eqnarray*}



Let us compute $f'(1)$ and $f'(-1)$. (We may assume $f'(-1),f'(1)\ne0$, since otherwise $T=0$ and the inequality $\frac{2}{3}\cdot T\le A$ becomes trivial). By (1) above we see
$$
f'(1)=-2\prod_i(\alpha_i-1)\prod_j(\beta_j+1),
$$
and similarity
$$
f'(-1)=2\prod_i(\alpha_i+1)\prod_j(\beta_j-1).
$$
Hence we conclude 
$$
A > \frac{2}{3}\left(-f'(1)f'(-1)\right)^\frac{1}{2}.
$$
Applying now the inequality of the harmonic and the geometric mean to $-f'(1)$ and $f'(1)$, we arrive at the conclusion
$$
A\ge\frac{2}{3}\cdot\frac{2}{\frac{1}{-f'(1)}+\frac{1}{f'(-1)}}=\frac{4}{3}\cdot\frac{f'(1)f'(-1)}{f'(1)-f'(-1)}=\frac{2}{3}\cdot T
$$


\newpage

For any positive integer $n$, if $n^2$ is divisible by 3, then $n$ is divisible by 3.

\newpage

Let $n$ be an integer such that $n^2=3x$, where $x$ is any integer. Then $3\, |\, n^2$. 
Since $n^2=3x$, $n\cdot n=3x$. Thus, $3\, |\, n$. Therefore if $n^2$ is a multiple of 3, then $n$ is a multiple of 3. 




% \bibliography{bibliography}
% \end{onehalfspace}
 \end{document} 